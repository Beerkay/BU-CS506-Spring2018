\documentclass[11pt]{article}
\pagestyle{empty}

\setlength{\textheight}{8.5in}
\setlength{\topmargin}{0.5in}
\setlength{\headheight}{0in}
\setlength{\headsep}{0in}
% %% \setlength{\footheight}{0in}
\setlength{\oddsidemargin}{0in}
\setlength{\textwidth}{6.5in}

\usepackage{times}
%\usepackage{url}
\usepackage{algorithm}
\usepackage{mathtools}
\usepackage{mathptmx}
\usepackage{amssymb}
\usepackage{color}
\usepackage{paralist}
\usepackage{ulem}
\normalem
\usepackage{hyperref}


\begin{document}
\sloppy 
\begin{center}
\LARGE CAS CS 506\\
\Large Computational Tools for Data Science\\
\Large\rm Fall 2017\\~\\
\end{center}

\noindent{\large\bf Lectures:} Tuesdays and Thursdays, 3:30--4:45
PM in Photonics 211

\noindent \textbf{Lecture schedule}: 
\url{https://tinyurl.com/cs506-spring2018}

\noindent \textbf{Piazza site:} \url{https://piazza.com/bu/spring2018/cs506}

\noindent \textbf{Lecture materials:} \url{https://github.com/adamdavisonsmith/BU-CS506-Spring2018}


\noindent{\large\bf Instructor:} Adam Smith
  \begin{compactitem}
  \item {\bf Office:} MCS-135F
  \item {\bf Office Hours:} {\small Tuesdays 11:00AM--12:30pm, Thursdays
      12:00PM--1:30PM}
  \item {\bf Email:} ads22@bu.edu (but use Piazza)
  \end{compactitem} 

\medskip


 \noindent{\large\bf Teaching Fellow:} Ms.\ Sofia Nikolakaki
 \begin{compactitem}
 \item {\bf Office Hours:} {\small TBD.}
 \item {\bf Office Hours Location:} Undergrad Lab, EMA 302
 \item {\bf Lab Tutoring Hours:} {\small TBD.}
 \item {\bf Email:} smnikol@bu.edu (but use Piazza)
 \end{compactitem}

\noindent{\large\bf Co-instructors (projects) :} Dora Erdos (\texttt{edori@bu.edu}), Ziba
Cranmer (\texttt{zcranmer@bu.edu})

\medskip


\noindent \textbf{Action items} you should complete today:
\begin{compactitem}
\item Read the syllabus
\item Send Sofia your github account name (you may need to create one)
\item Add yourself to the course Piazza site
\item Fill out the course survey (see Piazza for link)
\item Install python, git (see notes for Lecture 02)
\end{compactitem}






\section*{Overview of the Course}

This course is targeted at students who require a basic level of
proficiency in working with and analyzing data.  The course emphasizes
practical skills in working with data, while introducing students to a
wide range of techniques that are commonly used in the analysis of data,
such as clustering, classification, regression, and network analysis.
The goal of the class is to provide to students a hands-on understanding
of classical data analysis techniques and to develop proficiency in
applying these techniques in a modern programming language (Python). 

Broadly speaking, the course breaks down into three main components,
which we will take in order of increasing complication:  (a)
unsupervised methods and summaries; (b) supervised methods; and (c) methods for
structured data.

Lectures will present the fundamentals of each technique; the focus is 
% not
% on the theoretical analysis of the methods, but rather on 
helping
students (a) understand the practical settings in which these methods are
useful and (b) interpret the results (and assess the significance) of their analyses.
Class discussion will study use cases and will go over relevant
Python packages that will enable the students to perform hands-on
experiments with their data. 

\section*{Prerequisites} 
Students taking this class \textbf{must} have 
\begin{compactitem}
  \item \emph{Experience programming}, at the level of CS 105, 108, or 111, or
  equivalent.  
\item A solid understanding of linear algebra. CS 132 or equivalent (MA 242, MA 442) is
  \textbf{required}. In particular, students should be comfortable with the notions
  of linear independence, rank,
  eigenvalue and eigenvector.
\item \emph{Probability and statistics}. We will assume familiarity
  with basic concepts of probability (indendence, random variables, expectation,
  variance) and statistics (point estimates, regression, confidence intervals,
  hypothesis tests).
\item \emph{General scientific mathematics}. We will assume familiarity
  with basic concepts of probability (independence, random variables, expectation,
  variance) and statistics (point estimates, confidence intervals,
  hypothesis tests), as well as calculus.

\end{compactitem}
\noindent Other useful background:
\begin{compactitem}
\item Data structures and algorithms (CS 112, 131)
\item Probability for computer scientists (CS 337)
\item Vector calculus
\end{compactitem}

\section*{Learning Outcomes}

Students who successfully complete this course will be proficient in
basic data acquisition, manipulation, and analysis.  They will be able
to understand and carry out the most commonly used methods of clustering,
classification, and regression. They will be able to interpret their
results, and understand the limitations of their methods. 
   They will also understand and be able to articulate 
efficiency and systems issues related to working on very large
datasets. 

\section*{Readings} 

There is no text for the course.   Lecture notes will be posted online.

Some of the lectures are based on \emph{Introduction to Data Mining,} by
Tan, Steinbach and Kumar.  This is a good place to go for more detail if
something is not clear.

Some other recommended texts:
\begin{compactitem}
\item \emph{Python for Data Analysis}
  (\url{http://shop.oreilly.com/product/0636920023784.do}).  This is the
  definitive text for \emph{Pandas} which we will use quite a bit.

\item Larry Wasserman, \emph{All of Statistics: A Concise Course in Statistical
    Inference}, Springer,  2004. 

\item \emph{Programming Collective Intelligence}
  (\url{http://shop.oreilly.com/product/9780596529321.do})


\end{compactitem}

\section*{Web Resources} 

Many of the  slides for the course are actually executable python scripts, using the
\texttt{jupyter notebook}.   You can
download and execute the lectures on your own computer, and you can
modify them any way you'd like, play around with them, experiment, etc.

The slides I use in lecture are published on \texttt{github}.   The
repository is
\url{https://github.com/adamdavisonsmith/BU-CS506-Spring2018}.  If you want
to access the repository using \texttt{git}, please feel free.   If you
find a bug, submit a pull request (alternatively, ask
about it on Piazza.)
 
\section*{Homeworks and Project}

\begin{enumerate}
\item There will seven to nine homework assignments.  In a typical 
assignment you will 
analyze one or more datasets using the tools and techniques presented in
class.

Homeworks will be submitted via \texttt{github}.   For this, we
  need your github account (create one if you don't already have it).
  After you have created it, fill out the form at \textbf{TBD}
  (was \url{https://goo.gl/forms/8W0SOdvMn07UKdip2}) to let us know what it is.

You are expected to work individually on homeworks.

\item There will be a final project.  For the project you
  will extract some
knowledge or conclusions from the analysis of dataset of your choice. The analysis
will be done using a subset of the methods we described in class.  The
final project will require a proposal, two progress reports, and a final
presentation in poster form.

The project will have three essential components: 1) a data collection
piece (which may involve crawling or calls to an API, combining data
from different sources etc), 2) a data analysis piece (which will
involve applying different techniques we described in class for the
analysis) and 3) a conclusion component (where the results of the data
analysis will be drawn).  The students will submit a 5-page report
explaining clearly all the three components of their project. Finally a
poster presentation will be required where the students will be prepare
to present their effort and results in front of their poster. 

As an example, you may choose to collect data from Twitter related to
a specific topic (e.g., Ebola virus) and then measure the intensity of
posts about a topic in different areas of the world.  Other examples
of projects may include (but are not limited to): analysis of MBTA
data, analysis of census data from your favorite country, crawling of
YouTube (or other social media data) and analysis of social behavior
like trolling and bullying.

The main project report is due in early April; a revised report is due
at the end of April. See the lecture schedule page for exact dates. The project
presentations will be given in the form of a final poster explaining
components 1, 2 and 3 of the project.

You are expected to work in teams of two on the final project.   I will
leave it up to you to form teams on your own, but everyone must work in
a team.

\end{enumerate}

\section*{Piazza}

Piazza is a website that allows you to ask questions, either to
instructors or course-wide.  We will be using Piazza for almost all course
communication outside of the classroom. Please sign up, and set
appropriate email notification options so that you make sure to
receive announcements.

\url{https://piazza.com/bu/spring2018/cs506}. 

Piazza allows you to ask questions that are visible only to
instructors, but it also allows you to ask questions to the entire
class, and answer others' questions. When someone posts a question on
Piazza, if you know the answer, please go ahead and post it.  However
please \emph{do not} provide answers to homework questions on Piazza.
It's OK to tell people \emph{where to look} to get answers, or to
correct mistakes; just don't provide actual solutions to
homeworks. Also, be polite. See the post ``Ethics and Etiquette on Piazza'' 
for more detail.

\section*{Programming Environment}

We will use \texttt{python} as the language for teaching and for
assignments that require coding.    Instructions for installing and
using Python are on Piazza.

\section*{Course and Grading Administration}

Homeworks are due at 7pm on Fridays.
Assignments will be submitted using \texttt{github}.   Ms.\ Nikolakaki will
explain how to submit assignments.  

\emph{IMPORTANT:} Late assignments \textbf{WILL NOT} be accepted. 
However, you may submit \textbf{one} homework up to 3 days late.   You
\textbf{must} email Ms.\ Nikolakaki before the deadline if you intend to
submit a homework late. 

Final grades will be computed based on the following:
\begin{description}
\item[50\%] Homework assignments.  
\item[50\%] Final Project
\end{description}

The exact cutoffs for final grades will be determined after the class is
complete.



\section*{Academic Honesty}

You may discuss homework assignments with classmates, but you are 
solely responsible for what you turn in. Collaboration in the form of
discussion is allowed, but all forms of cheating (copying parts of a
classmate's assignment, plagiarism from books or old posted solutions)
are NOT allowed. We -- both teaching staff and students -- are expected
to abide by the guidelines and rules of the Academic Code of Conduct
at
\url{http://www.bu.edu/academics/policies/academic-conduct-code/}.

Graduate students must also be aware of and abide by the GRS Academic
Conduct code at \url{http://www.bu.edu/cas/students/graduate/forms-policies-procedures/academic-discipline-procedures/}.

You can probably, if you try hard enough, find solutions for homework
problems online. That said,
\begin{enumerate}
\item If you are looking online for an answer because you don't know how
  to start thinking about a problem, talk to Ms.\ Nikolakaki or myself, who may be
  able to give you pointers to get you started.  Piazza is great for
  this -- you can usually get an answer in an hour if not a few minutes.
\item If you are looking online for an answer because you want to see if
  your solution is correct, ask yourself if there is some way to verify
  the solution yourself.   Usually, there is.  You will understand what you have done
  \emph{much} better if you do that.
\item We will enforce the collaboration policy.
\end{enumerate}

\newpage
\section*{Course Schedule}

We will maintain a detailed lecture schedule here: 

\url{https://tinyurl.com/cs506-spring2018}


Topics (tentative):
\begin{compactitem}
\item Introduction
  \begin{compactitem}
\item Python and essential tools (Git, Jupyter Notebook, Pandas)
  \item The process of ``data science''
  \item Probability and Statistics Refresher
  \item Linear Algebra Refresher
  \end{compactitem}
\item A sampling of techniques
  \begin{compactitem}
\item Distance and Similarity Functions, Timeseries
  \item Clustering
    % \item Clustering I: k-means
    % \item Clustering II: In practice
    % \item Clustering III: Hierarchical Clustering
    % \item Clustering IV: GMM and Expectation Maximization
  \item Assessing significance: correlations and clustering
  \item Singular Value Decomposition and Dimension reduction
    % \item Singular Value Decomposition I : Low Rank Approximation
    % \item SVD II: Dimensionality Reduction
    % \item SVD III: Anomaly Detection
  \item Web Scraping
  \item Classification and Regression
  \end{compactitem}
\item Interpretation, assessment, and confidence
  \begin{compactitem}
  \item statistical validity
  \item p-hacking and multiple hypothesis testing
  \end{compactitem}
\item Ethics of data: privacy, transparency, accountability, representation, fairness

\item Advanced technical topics
  \begin{compactitem}
\item Parallel architectures and Map Reduce

  \item Collaborative filtering and recommender Systems


  \item Analyzing Graphs and Networks
  \end{compactitem}
% \item Network Analysis IA
% \item Network Analysis IB
% \item Networks II: Centrality 
% \item Networks III: Clustering
\end{compactitem}



\end{document}

